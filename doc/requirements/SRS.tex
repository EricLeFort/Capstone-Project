\documentclass[titlepage]{article}
\usepackage[left=15mm,right=15mm,top=1in,bottom=1in]{geometry}
\usepackage{framed}
\usepackage{imakeidx}
\usepackage{graphicx}
\usepackage{array}

\newcolumntype{C}[1]{>{\centering\arraybackslash} m{#1cm}}

\makeindex

\title{Project Title\\~\\Software Requirements Specification}
\author{
Eric Le Fort\\leforte@mcmaster.ca\\1308609\and
Max Moore\\moorem8@mcmaster.ca\\1320009
}

\begin{document}
\maketitle


\tableofcontents
\listoffigures
\listoftables


\vfill
\begin{table}[!htbp]
\centering
\begin{tabular}{| C{3} | C{2} | C{5} | C{2.5} |}\hline
	Date			&Revision \#	&Comments						&Authors\\\hline
	DD/MM/YYYY		&0				&- Initial document creation	&~Eric Le Fort\newline Max Moore\\\hline
\end{tabular}
\caption{Revision History}
\end{table}
\newpage


\section{Project Drivers}
\subsection{The Purpose of the Project}
%TODO
The aim of this project is to create an automated pool playing robot. This robot will be able to play pool against a
human opponent for recreation or training purposes as determined by the user.
What follows is a breakdown of what the system will do step-by-step in more specific detail. To initiate the robot's
turn, the user will press a button signifying that the system should begin. A camera will then be used to view the table
and, using visual recognition algorithms, map out the positions of the pool balls. Our system will then determine the best
angle at which to take a shot and how to move the equipment to that position. Once that is done, the robot will move
into place, lining up the shot. Using a specialized, built-in pool cue, the robot will then make its shot.
Once it is the user's turn, the user will also be able to press a button to indicate to the robot that it needs to move in
order to give room for the human player's shot. The robot will then move to a position that is out of the way.


\subsection{The Clients, the Customers, and Other Stakeholders}

\subsubsection{The Clients}
%TODO


\subsubsection{The Customers}
%TODO

\subsubsection{Other Stakeholders}
%TODO

\subsection{Users of the Product}
%TODO


\section{Project Constraints}
\subsection{Mandated Constraints}
%TODO

\subsection{Naming Conventions \& Definitions}
This section outlines the various definitions, acronyms and abbreviations that will be used throughout this document in order to familiarize the reader prior to reading.
\subsubsection{Definitions}
Table \ref{tab:Definitions} lists the definitions used in this document. The definitions given below are specific to this document and may not be identical to definitions of these terms in common use. The purpose of this section is to assist the user in understanding the requirements for the system.
\begin{table}[h!]
\centering
\caption{Definitions}
    \begin{tabular}{| p{6cm} | p{6cm} |}    \hline
    \textbf{Term} &\textbf{Meaning}\\ \hline
    %TODO Fill with necessary definitions.
    \end{tabular}
\label{tab:Definitions}
\end{table}

\subsubsection{Acronyms \& Abbreviations}
Table \ref{tab:Acronyms} lists the acronyms and abbreviations used in this document.
\begin{table}[h!]
\centering
\caption{Acronyms and Abbreviations}
    \begin{tabular}{| p{6cm} | p{6cm} |}    \hline
    \textbf{Acronym/Abbreviation} &\textbf{Meaning}\\ \hline
    %TODO Fill with necessary acronyms/abbreviations.
    \end{tabular}
\label{tab:Acronyms}
\end{table}

\subsection{Relevant Facts \& Assumptions}
%TODO


\section{Functional Requirements}
\subsection{The Scope of the Work}
%TODO

\subsection{The Scope of the Product}
%TODO

\subsection{Functional \& Data Requirements}
\begin{framed}
\noindent\textbf{Requirement \#}: - \hfill \textbf{Requirement Type}: - \hfill\\\\
\noindent\textbf{Description}: -\\
\textbf{Rationale}: -\\
\textbf{Originator}: (Author)\\
\textbf{Fit Criterion}: -\\\\
\noindent\textbf{Customer Satisfaction}: (0-5) \hfill 	\textbf{Customer Dissatisfaction}: (0-5) \hfill\\
\textbf{Priority}: (low/medium/high) \hfill \textbf{Conflicts}: () \hfill 		\\
\textbf{Supporting Material}: (Other documents)\\\\
\noindent\textbf{History}: Created (DATE)
\end{framed}


\section{Non-Functional Requirements}
\subsection{Look \& Feel Requirements}
\textbf{LF1:} %TODO

\subsection{Usability \& Humanity Requirements}
\textbf{UH1:} %TODO

\subsection{Performance Requirements}
\textbf{PR1:} %TODO

\subsection{Operational \& Environmental Requirements}
\textbf{OE1:} %TODO

\subsection{Maintainability \& Support Requirements}
\textbf{MS1:} %TODO

\subsection{Security Requirements}
\textbf{S1:} %TODO

\subsection{Cultural \& Political Requirements}
\textbf{CP1:} %TODO

\subsection{Legal Requirements}
\textbf{L1:} %TODO


\section{Project Issues}
\subsection{Open Issues}
%TODO

\subsection{Off-the-Shelf Solutions}
%TODO

\subsection{New Problems}
%TODO

\subsection{Tasks}
%TODO

\subsection{Migration to the New Project}
%TODO

\subsection{Risks}
%TODO

\subsection{Costs}
%TODO

\subsection{User Documentation \& Training}
%TODO

\subsection{Waiting Room}
%TODO

\subsection{Ideas for Solutions}
%TODO

\pagebreak
\printindex
\end{document}