\documentclass[titlepage]{article}
\usepackage[left=15mm,right=15mm,top=1in,bottom=1in]{geometry}
\usepackage{hyperref}
\usepackage{color}
\usepackage{tabularx}

\newcolumntype{C}[1]{>{\centering\arraybackslash} m{#1cm}}

\title{Autonomous Pool Playing Robot\\~\\\textbf{Requirements Specification}}
\author{
	%Ernest Selman\\selmae@mcmaster.ca\\1201291\\~\\\and
	Eric Le Fort\\leforte@mcmaster.ca\\1308609%\\~\\\and
	%Guy Meyer\\meyerg@mcmaster.ca\\1320231\\~\\\and
	%Andrew Danha\\danhaas@mcmaster.ca\\1223881\\~\\\and
	%Max Moore\\moorem8@mcmaster.ca\\1320009\\~\\\and
	%Derek Savery\\saverydj@mcmaster.ca\\1219142\\~\\
}
\begin{document}
\maketitle
\tableofcontents
~\\[15mm]
\listoftables

%TODO include:
%Testing policy
%Details of test factors used
%Formal proof (good for confirming test coverage)
%Define/use test coverage criteria
%PASS/FAIL criteria for each test

%TODO Actually having ran tests is marked..

\vfill
\begin{table}[!htbp]
\centering
\begin{tabular}{| C{3} | C{2} | C{5} | C{2.5} |}\hline
	Date			&Revision \#	&Comments						&Authors\\\hline
	27/02/2017		&0				&- Initial document creation	&Eric Le Fort\\\hline
\end{tabular}
\caption{Revision History}
\end{table}
\newpage

\section{Introduction}
This document will provide a specification of a test plan for an automated pool playing robot and report on the results of that plan.
\subsection{Overview}
This document breaks down the required testing for each domain of the system. It begins with the hardware aspect, then moves to the electrical side and then finishes with software. Each section will provide a traceability matrix to map the requirements to tests that check their completion and then go into further detail to describe each test case. Lastly, a summary of the results of testing will be provided to conclude the document.
\subsection{Purpose}
The aim of this document is to illuminate any design flaws, software bugs, or other issues in the system. Once these issues are discovered, the engineering team will be able to work on eliminating them or minimizing their frequency and consequences.
\subsection{Naming Conventions \& Definitions}
This section outlines the various definitions, acronyms and abbreviations that will be used throughout this document in order to familiarize the reader prior to reading.
\newpage

\subsubsection{Definitions}
Table \ref{tab:Definitions} lists the definitions used in this document. The definitions given below are specific to this document and may not be identical to definitions of these terms in common use. The purpose of this section is to assist the user in understanding the requirements for the system.
\begin{table}[h!]
\centering
\caption{Definitions}
\begin{tabular}{| C{6} | p{6cm} |}\hline
	\textbf{Term}	&\textbf{\centering Meaning}\\\hline
	X-axis					&Distance along the length of the pool table\\\hline
	Y-axis					&Distance across the width of the pool table\\\hline
	Z-axis					&Height above the pool table\\\hline
	End-effector			&The end of the arm that will strike the cue ball\\\hline
	$\theta$				&Rotational angle of end-effector\\\hline
	Cue 					&End-effector\\\hline
	Personal Computer		&A laptop that will be used to run the more involved computational tasks such as visual recognition and the shot selection algorithm\\\hline
	Camera					&Some form of image capture device (e.g. a digital camera, smartphone with a camera, etc.)\\\hline
	Table State				&The current positions of all the balls on the table\\\hline
	Entity					&Classes that have a state, behaviour and identity (e.g. Book, Car, Person, etc.)\\\hline
	Boundary				&Classes that interact with users or external systems\\\hline
	Double					&Double-precision floating point numbers\\\hline
\end{tabular}
\label{tab:Definitions}
\end{table}
\subsubsection{Acronyms \& Abbreviations}
Table \ref{tab:Acronyms} lists the acronyms and abbreviations used in this document.
\begin{table}[h!]
\centering
\caption{Acronyms and Abbreviations}
\begin{tabular}{| p{6cm} | p{6cm} |}\hline
	\textbf{Acronym/Abbreviation}	&\textbf{Meaning}\\\hline
	VR								&Visual Recognition\\\hline
	PC								&Personal Computer\\\hline
	$\mu$C							&Micro-Controller\\\hline
	CRC								&Class Responsibility Collaboration\\\hline
\end{tabular}
\label{tab:Acronyms}
\end{table}

\section{Mechanical Components}
%TODO Traceability Matrix
%TODO acceleration/strike force
%TODO range of motion/table interference avoidance
%TODO speed
%TODO anything else?


\section{Electrical System}
%TODO Traceability Matrix
%TODO correct Amp/Volt
%TODO latency checks
%TODO anything else?

\section{Software System}
The software system is comprised of four main components: a control system running on an Arduino microcontroller, an automated image capture application running on an Android smartphone, as well a visual recognition program and smart shot selection program running on a PC. On top of the typical suite of unit tests to verify correctness of methods, rigorous system testing will also be crucial to adequately test this system.\\~\\
The following traceability matrix will demonstrate that the tests to be performed prove that specified requirements have been met.
%TODO Traceability Matrix
\subsection{Unit Tests}
%TODO describe what's considered a pass/fail for each test
\subsubsection{PC Controller Program}
%TODO F2, F4, F5, F7, F8, F17
\subsubsection{PC VR Program}
%TODO F3, F5, F17
\subsubsection{$\mu$C Program}
%TODO F6-15, UH1, UH3
\subsubsection{Android Program}
%TODO F1, F2
%TODO Tables with following columns: Test ID, Inputs, Expected Values, Actual Values, Pass/Fail
\subsection{System Tests}
%TODO P1, P2, S2(?), S6, CP1, L1
%TODO functionality/accuracy
%TODO performance
%TODO explain testing methodology where necessary

\begin{center}
\begin{table}
\begin{tabular}{|l r|}\hline&\\[-2mm]
	Test ID: n	&Status: TBT\\[-3mm]
	\multicolumn{2}{|c|}{Module: ModuleName}\\&\\
	\multicolumn{2}{|c|}{\textbf{\large{TestTitle}}}\\&\\\hline&\\[-3mm]
	\multicolumn{2}{|p{\textwidth}|}{Pass/Fail Conditions:}\\[1mm]\hline&\\[-3mm]
	\multicolumn{2}{|p{\textwidth}|}{Pre-Conditions:}\\[4mm]
	\multicolumn{2}{|p{\textwidth}|}{Input:}\\[2mm]\hline
	\multicolumn{1}{|p{0.49\textwidth}}{Expected Results: }	&\multicolumn{1}{|p{0.45\textwidth}|}{Actual Results: }\\\hline&\\[-3mm]
	\multicolumn{2}{|p{\textwidth}|}{Post-Conditions: }\\\hline
\end{tabular}
\caption{Test Title}
\end{table}
\end{center}

\section{Summary of Results}
%TODO overall satisfaction with results
%TODO relate back to requirements, goals

\end{document}



%%% TEST CASE TEMPLATE %%%
\begin{center}
\begin{table}
\begin{tabular}{|l r|}\hline&\\[-2mm]
	Test ID: n	&Status: TBT\\[-3mm]
	\multicolumn{2}{|c|}{Module: ModuleName}\\&\\
	\multicolumn{2}{|c|}{\textbf{\large{TestTitle}}}\\&\\\hline&\\[-3mm]
	\multicolumn{2}{|p{\textwidth}|}{Pass/Fail Conditions:}\\[1mm]\hline&\\[-3mm]
	\multicolumn{2}{|p{\textwidth}|}{Pre-Conditions:}\\[4mm]
	\multicolumn{2}{|p{\textwidth}|}{Input:}\\[2mm]\hline
	\multicolumn{1}{|p{0.49\textwidth}}{Expected Results: }	&\multicolumn{1}{|p{0.45\textwidth}|}{Actual Results: }\\\hline&\\[-3mm]
	\multicolumn{2}{|p{\textwidth}|}{Post-Conditions: }\\\hline
\end{tabular}
\caption{Test Title}
\end{table}
\end{center}







