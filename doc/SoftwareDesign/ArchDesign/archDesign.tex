\documentclass[titlepage]{article}
\usepackage[left=15mm,right=15mm,top=1in,bottom=1in]{geometry}
\usepackage{framed}
\usepackage{imakeidx}
\usepackage{graphicx}
\usepackage{array}

\newcolumntype{C}[1]{>{\centering\arraybackslash} m{#1cm}}

\makeindex

\title{Autonomous Pool Playing Robot\\~\\High-Level Architectural Design}
\author{
	Eric Le Fort\\leforte@mcmaster.ca\\1308609\\~\\\and
	Max Moore\\moorem8@mcmaster.ca\\1320009
}
 
\begin{document}
\maketitle
\tableofcontents
~\\[15mm]
\listoftables


\vfill
\begin{table}[!htbp]
\centering
\begin{tabular}{| C{3} | C{2} | C{5} | C{2.5} |}\hline
	Date			&Revision \#	&Comments						&Authors\\\hline
	14/11/2016		&0				&- Initial document creation	&Eric Le Fort\newline Max Moore\\\hline
\end{tabular}
\caption{Revision History}
\end{table}
\newpage
 
\section{Introduction}
This document's purpose is to describe the architecture of the software controlling the Autonomous Pool Playing Robot. Both the architecture for the encompassing system as well as architectures for distinct subsystems will be discussed.
\subsection{System Description}
This system will provide a control system for an autonomous pool playing robot. It will include three separate computational units and four separate programs.\\~\\
The first computational unit will be a camera (likely a camera phone). This device will be responsible for reacting to a request, taking an image and then relaying that image.\\~\\
The next computational unit will be the PC. This device will have two separate programs that must execute for the system to be successful. One program will handle the VR and the other will handle shot selection as well as message passing between devices. The VR program will handle processing the image from the camera into a table state that can be used by the shot selection algorithm. The other program on the PC will handle performing an algorithm to determine which shot should be taken, where to move the cue in order to take that shot, instructing the camera to take pictures, receiving the image from the camera, receiving a signal to take a shot from the $\mu$C and communicating the shot that must be taken back to the $\mu$C.\\~\\
The last computational unit, the $\mu$C, will be responsible for interpreting the shot instructions provided by the PC into signals to control the machine accordingly, receiving control signals from the system, providing signals to move the machine out of the way to predetermined locations when requested, and sending the signal to take a shot back to the PC.
\subsection{Overview}
This document has four sections not including this one. Each section contains either design diagrams or further explanations to further describe the architecture of this system and is intended to prepare the software team to implement the design.\\
\begin{itemize}
	\item \textbf{Use Cases}: Describes possible user interactions with the system as well as the intended results of those interactions through the use of simple diagrams. A Use Case Diagram is also provided for reference.\\
	\item \textbf{Analysis Class Diagram}: Defines the various classes in the system, how they will be connected, and their type (boundary, controller, or entity).\\
	\item \textbf{Architectural Design}: This section defines the overall system architecture as well as the architectures of all subprograms.\\
	\item \textbf{CRC Cards}: Each program will be broken up into their specific classes. The responsibilities of each class as well as any collaboration required with other classes to fulfill each responsibility (if any) will be listed.\\
\end{itemize}

\subsection{Naming Conventions \& Definitions}
This section outlines the various definitions, acronyms and abbreviations that will be used throughout this document in order to familiarize the reader prior to reading.
\subsubsection{Definitions}
Table \ref{tab:Definitions} lists the definitions used in this document. The definitions given below are specific to this document and may not be identical to definitions of these terms in common use. The purpose of this section is to assist the user in understanding the requirements for the system.
\begin{table}[h!]
\centering
\caption{Definitions}
\begin{tabular}{| C{6} | p{6cm} |}\hline
	\textbf{Term}	&\textbf{\centering Meaning}\\\hline
	X-axis					&Distance along the length of the pool table\\\hline
	Y-axis					&Distance across the width of the pool table\\\hline
	Z-axis					&Height above the pool table\\\hline
	End-effector			&The end of the arm that will strike the cue ball\\\hline
	$\theta$				&Rotational angle of end-effector\\\hline
	Cue 					&End-effector\\\hline
	Personal Computer		&A laptop that will be used to run the more involved computational tasks such as visual recognition and the shot selection algorithm\\\hline
	Camera					&Some form of image capture device (e.g. a digital camera, smartphone with a camera, etc.)\\\hline
	Table State				&The current positions of all the balls on the table\\\hline
	Entity					&Classes that have a state, behaviour and identity (e.g. Book, Car, Person, etc.)\\\hline
	Boundary				&Classes that interact with users or external systems\\\hline
\end{tabular}
\label{tab:Definitions}
\end{table}

\newpage
\subsubsection{Acronyms \& Abbreviations}
Table \ref{tab:Acronyms} lists the acronyms and abbreviations used in this document.
\begin{table}[h!]
\centering
\caption{Acronyms and Abbreviations}
\begin{tabular}{| p{6cm} | p{6cm} |}\hline
	\textbf{Acronym/Abbreviation}	&\textbf{Meaning}\\\hline
	VR								&Visual Recognition\\\hline
	PC								&Personal Computer\\\hline
	$\mu$C							&Micro-Controller\\\hline
	CRC								&Class Responsibility Collaboration\\\hline
\end{tabular}
\label{tab:Acronyms}
\end{table}


\section{Use Cases}
This section will outline the use cases that this system will be expected to handle. A Use Case Diagram will also be provided to help illustrate these cases. All use cases are initiated by the user interacting with a physical interface that will send a signal to the $\mu$C. Between operations, the various programs will revert to a dormant state to await the next instruction.
\subsection{Use Case Diagram}
The following diagram depicts the use cases of this system.
%\includegraphics[width=\textwidth]{useCaseDiagram.png}
%TODO
\subsection{Move Instruction}
This use case involves only the $\mu$C. Once the signal has been received, the $\mu$C will decide which predetermined set point to move to based on the current position of the machine that is stored in its memory. The $\mu$C will then determine the signals required to relocate the machine to the appropriate location and then send them to the machine. Once those signals are sent, the $\mu$C will update its memory to store the new position of the machine.
\subsection{Cancel Instruction}
This use case involves only the $\mu$C. Once the signal has been received, the $\mu$C will immediately stop the machine's motion. It will then update its memory to store the current position of the machine.
\subsection{Take a Shot Instruction}
This use case involves all four programs: the $\mu$C, the PC Controller Program, the Camera Program, and the PC VR Program in that order. Once the signal has been received, the $\mu$C will send a signal to the PC to begin its process and wait a small amount of time for the PC to send a receipt response. If the $\mu$C does not receive a response within the specified timeframe, it will resend its response. This process will be repeated until a response is received. The PC Controller Program will then send a signal to the camera to take a photo. Upon receiving this signal, the camera will take a picture of the table and send this back to the PC. If the picture was received correctly, the PC Controller Program will then start the PC VR Program and provide the picture to be used. Otherwise, the PC Controller Program will continue to resend the request to take a picture as necessary. The PC VR Program will utilize object detection to determine where the balls are on the table and which ball is which. Once collecting this information, it is provided to the PC Controller Program through an intermediate file. The PC Controller Program will then use its shot selection algorithm to determine how to strike the pool ball to take based on the table state provided and transmit that shot to the $\mu$C. After that, the $\mu$C will send a receipt signal back to the PC Controller Program. The $\mu$C will interpret the shot information to create an instruction set of signals to move the machine appropriately and take the shot.


\section{Analysis Class Diagrams}
%TODO create these diagrams in draw.io depending on how complicated they get
\subsection{Camera Analysis Class Diagram}
%TODO 
\subsection{PC Controller Analysis Class Diagram}
%TODO
\subsection{PC VR Program Analysis Class Diagram}
%TODO
\subsection{$\mu$C Analysis Class Diagram}
%TODO


\section{Architectural Design}
This section will discuss the architectures that are to be used while designing this system. The goal of these architectures is to promote the overall quality of the software for this system. In particular, the software should be more efficient, robust, and maintainable as a result of this architectural design.
\subsection{System Architecture}
The architecture of this system as a whole will be modelled using the batch sequential style. This architecture involves sending chunks of data after the program is completely done creating and promotes simple division of subsystems that can operate as independent programs. This creates a very strong separation of concerns within the larger system.\\~\\
The shortcomings of this system are that it does not allow for concurrency and it has low throughput. Luckily, in this situation, the throughput is not intended to be very high. Also, each subsystem must wait for the previous system to fully complete its task in order to proceed regardless. Therefore, since this system does not even allow for concurrent operation, there is no loss in potential efficiency.
\subsection{Subsystems}
The subsystems will follow the batch sequential architecture on a system level but that architecture does not make sense for the subsystems themselves. This section will discuss the architectures that will be utilized on the subsystem level.
\subsubsection{Camera Architecture}
%TODO Event-Driven
\subsubsection{PC Controller Architecture}
%TODO Rule-Based
\subsubsection{PC VR Program Architecture}
%TODO Pipe & Filter
\subsubsection{$\mu$C Architecture}
%TODO Sensor-Controller-Actuator


\section{Class Responsibility Collaboration (CRC) Cards}
%TODO
\subsection{Camera CRC Cards}
%TODO
\subsection{PC Controller CRC Cards}
%TODO
\subsection{PC VR Program CRC Cards}
%TODO
\subsection{$\mu$C CRC Cards}
%TODO


\pagebreak
\printindex
\end{document}

\begin{table}[!htbp]
\centering
\begin{tabular}{| p{0.6\textwidth} | p{0.3\textwidth} |} \hline 
	\multicolumn{2}{|l|}{\textbf{Class Name: -}} \\\hline
	\textbf{Responsibility:} & \textbf{Collaborators:} \\\hline
	
\end{tabular}
\end{table}



