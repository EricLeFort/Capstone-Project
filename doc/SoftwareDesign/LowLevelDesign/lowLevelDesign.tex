\documentclass[titlepage]{article}
\usepackage[left=15mm,right=15mm,top=1in,bottom=1in]{geometry}
\usepackage{framed}
\usepackage{caption}
\usepackage{amsmath}
\usepackage{multicol}
\usepackage{color}
\usepackage{graphicx}
\usepackage{array}
\usepackage[figuresleft]{rotating}

\newcolumntype{C}[1]{>{\centering\arraybackslash} m{#1cm}}
\graphicspath{{./img/}}

\setlength{\columnseprule}{1pt}
\def\columnseprulecolor{\color{black}}

\makeindex

\title{Autonomous Pool Playing Robot\\~\\Low-Level Software Design}
\author{
	Eric Le Fort\\leforte@mcmaster.ca\\1308609\\~\\\and
	Max Moore\\moorem8@mcmaster.ca\\1320009
}
 
\begin{document}
\maketitle
\tableofcontents
\listoftables
\listoffigures


\vfill
\begin{table}[!htbp]
\centering
\begin{tabular}{| C{3} | C{2} | C{5} | C{2.5} |}\hline
	Date			&Revision \#	&Comments						&Authors\\\hline
	25/12/2016		&0				&- Initial document creation	&Eric Le Fort\\\hline
\end{tabular}
\caption{Revision History}
\end{table}
\newpage
 
\section{Introduction}
This document will outline the low-level software design for a autonomous pool-playing robot. The purpose of this document will be to document the decisions made concerning the system's design as well as provide enough detail so that the programming of the system can be as trivial as possible.
\subsection{System Description}
A system description can be found in the /textit{High-Level Software Design} document for this system.
\subsection{Overview}
This document will begin by providing a detailed class diagram of the classes in the system. Then, each module will be covered in more detail such as the module's responsibilities, secrets, Interface Specification (MIS), and Internal Design (MID). Lastly, the document will discuss the scheduling of tasks and provide state charts and sequence diagrams to help illustrate the scheduling.
\subsection{Naming Conventions \& Definitions}
This section outlines the various definitions, acronyms and abbreviations that will be used throughout this document in order to familiarize the reader prior to reading.
\subsubsection{Definitions}
Table \ref{tab:Definitions} lists the definitions used in this document. The definitions given below are specific to this document and may not be identical to definitions of these terms in common use. The purpose of this section is to assist the user in understanding the requirements for the system.
\begin{table}[h!]
\centering
\caption{Definitions}
\begin{tabular}{| C{6} | p{6cm} |}\hline
	\textbf{Term}	&\textbf{\centering Meaning}\\\hline
	X-axis					&Distance along the length of the pool table\\\hline
	Y-axis					&Distance across the width of the pool table\\\hline
	Z-axis					&Height above the pool table\\\hline
	End-effector			&The end of the arm that will strike the cue ball\\\hline
	$\theta$				&Rotational angle of end-effector\\\hline
	Cue 					&End-effector\\\hline
	Personal Computer		&A laptop that will be used to run the more involved computational tasks such as visual recognition and the shot selection algorithm\\\hline
	Camera					&Some form of image capture device (e.g. a digital camera, smartphone with a camera, etc.)\\\hline
	Table State				&The current positions of all the balls on the table\\\hline
	Entity					&Classes that have a state, behaviour and identity (e.g. Book, Car, Person, etc.)\\\hline
	Boundary				&Classes that interact with users or external systems\\\hline
\end{tabular}
\label{tab:Definitions}
\end{table}
\newpage

\subsubsection{Acronyms \& Abbreviations}
Table \ref{tab:Acronyms} lists the acronyms and abbreviations used in this document.
\begin{table}[h!]
\centering
\caption{Acronyms and Abbreviations}
\begin{tabular}{| p{6cm} | p{6cm} |}\hline
	\textbf{Acronym/Abbreviation}	&\textbf{Meaning}\\\hline
	VR								&Visual Recognition\\\hline
	PC								&Personal Computer\\\hline
	$\mu$C							&Micro-Controller\\\hline
	CRC								&Class Responsibility Collaboration\\\hline
\end{tabular}
\label{tab:Acronyms}
\end{table}

\section{Detailed Class Diagram}~\\
\begin{center}
	\includegraphics[width=\textwidth]{DetailedClassDiagram.png}
\captionof{figure}{The system's detailed class diagram.}
\label{fig:detailedClassDiagram}
\end{center}
\newpage

\section{Module Guide}
This section discusses the various modules that this system is comprised of. The modules are divided based on which program they belong to. For each module, its responsibilities, secrets, MIS, and MID will be outlined.
\subsection{Camera Modules}
The following is the module contained within the Camera subsystem.
\subsubsection{EventHandler}
\textbf{Responsibilities}\\
\\\\%TODO
\textbf{Secrets}\\
\\\\%TODO
\textbf{MIS}\\
\\\\%TODO
\textbf{MID}\\
\\\\%TODO
\subsection{PC VR Program Modules}
The following is the module contained within the PC VR subsystem.
\subsubsection{TableStateVR}
\textbf{Responsibilities}\\
\\\\%TODO
\textbf{Secrets}\\
\\\\%TODO
\textbf{MIS}\\
\\\\%TODO
\textbf{MID}\\
\\\\%TODO
\subsection{PC Controller Modules}
The following are the modules contained within the PC Controller subsystem.
\subsubsection{InferenceEngine}
\textbf{Responsibilities}\\
\\\\%TODO
\textbf{Secrets}\\
\\\\%TODO
\textbf{MIS}\\
\\\\%TODO
\textbf{MID}\\
\\\\%TODO
\subsubsection{PCCommunicator}
\textbf{Responsibilities}\\
\\\\%TODO
\textbf{Secrets}\\
\\\\%TODO
\textbf{MIS}\\
\\\\%TODO
\textbf{MID}\\
\\\\%TODO
\subsubsection{SimulationInstance}
\textbf{Responsibilities}\\
\\\\%TODO
\textbf{Secrets}\\
\\\\%TODO
\textbf{MIS}\\
\\\\%TODO
\textbf{MID}\\
\\\\%TODO
\subsection{$\mu$C Modules}
The following are the modules contained within the $\mu$C subsystem.
\subsubsection{Controller}
\textbf{Responsibilities}\\
\\\\%TODO
\textbf{Secrets}\\
\\\\%TODO
\textbf{MIS}\\
\\\\%TODO
\textbf{MID}\\
\\\\%TODO
\subsubsection{SensorMonitor}
\textbf{Responsibilities}\\
\\\\%TODO
\textbf{Secrets}\\
\\\\%TODO
\textbf{MIS}\\
\\\\%TODO
\textbf{MID}\\
\\\\%TODO
\subsubsection{ShotInterpreter}
\textbf{Responsibilities}\\
\\\\%TODO
\textbf{Secrets}\\
\\\\%TODO
\textbf{MIS}\\
\\\\%TODO
\textbf{MID}\\
\\\\%TODO
\subsubsection{$\mu$CCommunicator}
\textbf{Responsibilities}\\
\\\\%TODO
\textbf{Secrets}\\
\\\\%TODO
\textbf{MIS}\\
\\\\%TODO
\textbf{MID}\\
\\\\%TODO
\newpage

\section{Scheduling of Tasks}
The goal of this section is to outline the ordering, maximum allowable time frames, and the prioritization of tasks in this program.
%TODO Describe scheduling, especially timing constraints and the like.
\subsection{State Charts}
The following charts illustrate the lifecycle of all relevant classes in this system. This section is meant to depict a more isolated picture of how each class will operate.\\[15mm]
\begin{center}
	\includegraphics[width=0.6\textwidth]{ControllerStateChart.png}
\captionof{figure}{A state chart for the Controller class.}
\label{fig:ControllerStateChart}
\end{center}
\newpage

\begin{center}
	\includegraphics[width=0.8\textwidth]{ShotInterpreterStateChart.png}
\captionof{figure}{A state chart for the ShotInterpreter class.}
\label{fig:ShotInterpreterStateChart}
\end{center}
\begin{multicols}{2}
~\vfill
\begin{center}
	\includegraphics[width=0.4\textwidth]{SensorMonitorStateChart.png}
\captionof{figure}{A state chart for the SensorMonitor class.}
\label{fig:SensorMonitorStateChart}
\end{center}
~\vfill
\begin{center}
	\includegraphics[width=0.35\textwidth]{uCCommunicatorStateChart.png}
\captionof{figure}{A state chart for the $\mu$CCommunicator class.}
\label{fig:uCCommunicatorStateChart}
\end{center}
\end{multicols}
\newpage

~\vfill
\begin{center}
	\includegraphics[width=0.9\textwidth]{PCCommunicatorStateChart.png}
\captionof{figure}{A state chart for the PCCommunicator class.}
\label{fig:PCCommunicatorStateChart}
\end{center}
~\vfill
\newpage

~\vfill
\begin{center}
	\includegraphics[width=0.6\textwidth]{InferenceEngineStateChart.png}
\captionof{figure}{A state chart for the InferenceEngine class.}
\label{fig:InferenceEngineStateChart}
\end{center}
~\vfill
\newpage

\begin{multicols}{2}
\begin{center}
	\includegraphics[width=0.175\textwidth]{TableStateVRStateChart.png}
\captionof{figure}{A state chart for the TableStateVR class.}
\label{fig:TableStateVRStateChart}
\end{center}
\columnbreak
~\vspace{15mm}
\begin{center}
	\includegraphics[width=0.3\textwidth]{EventHandlerStateChart.png}
\captionof{figure}{A state chart for the EventHandler class.}
\label{fig:EventHandlerStateChart}
\end{center}
\end{multicols}
\newpage

\subsection{Sequence Diagrams}
The following are various sequence diagrams for different actions the system is required to perform. These diagrams are meant to provide better context for how the classes interact with each other to perform certain tasks.\vfill
\begin{center}
	\includegraphics[width=0.6\textwidth]{MoveSequenceDiagram.png}
\captionof{figure}{A sequence diagram for the ``move'' operation.}
\label{fig:MoveSequenceDiagram}
\end{center}
~\vfill
\begin{center}
	\includegraphics[width=0.6\textwidth]{cancelSequenceDiagram.png}
\captionof{figure}{A sequence diagram for the ``cancel'' operation.}
\label{fig:CancelSequenceDiagram}
\end{center}
~\vfill
\newpage

\begin{sidewaysfigure}
	\begin{center}
		\includegraphics[width=\textwidth]{TakeShotSequenceDiagram.png}
	\captionof{figure}{A sequence diagram for the ``take shot'' operation.}
	\label{fig:TakeShotSequenceDiagram}
	\end{center}
\end{sidewaysfigure}

\end{document}

