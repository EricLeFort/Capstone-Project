\documentclass[titlepage]{article}
\usepackage[left=15mm,right=15mm,top=1in,bottom=1in]{geometry}
\usepackage{framed}
\usepackage{caption}
\usepackage{amsmath}
\usepackage{multicol}
\usepackage{color}
\usepackage{graphicx}
\usepackage{array}
\usepackage{listings}
\usepackage[figuresleft]{rotating}

\newcolumntype{C}[1]{>{\centering\arraybackslash} m{#1cm}}
\graphicspath{{./img/}}

\setlength{\columnseprule}{1pt}
\def\columnseprulecolor{\color{black}}

\makeindex

\title{Autonomous Pool Playing Robot\\~\\Low-Level Software Design}
\author{
	Eric Le Fort\\leforte@mcmaster.ca\\1308609
}
 
\begin{document}
\maketitle
\tableofcontents
\listoftables
\listoffigures


\vfill
\begin{table}[!htbp]
\centering
\begin{tabular}{| C{3} | C{2} | C{5} | C{2.5} |}\hline
	Date			&Revision \#	&Comments						&Authors\\\hline
	25/12/2016		&0				&- Initial document creation	&Eric Le Fort\\\hline
	14/01/2017		&0				&- First draft completion		&Eric Le Fort\\\hline
\end{tabular}
\caption{Revision History}
\end{table}
\newpage
 
\section{Introduction}
This document will outline the low-level software design for a autonomous pool-playing robot. The purpose of this document will be to document the decisions made concerning the system's design as well as provide enough detail so that the programming of the system can be as trivial as possible.
\subsection{System Description}
A system description can be found in the /textit{High-Level Software Design} document for this system.
\subsection{Overview}
This document will begin by providing a detailed class diagram of the classes in the system. Then, each module will be covered in more detail such as the module's responsibilities, secrets, Interface Specification (MIS), and Internal Design (MID). Lastly, the document will discuss the scheduling of tasks and provide state charts and sequence diagrams to help illustrate the scheduling.
\subsection{Naming Conventions \& Definitions}
This section outlines the various definitions, acronyms and abbreviations that will be used throughout this document in order to familiarize the reader prior to reading.
\newpage

\subsubsection{Definitions}
Table \ref{tab:Definitions} lists the definitions used in this document. The definitions given below are specific to this document and may not be identical to definitions of these terms in common use. The purpose of this section is to assist the user in understanding the requirements for the system.
\begin{table}[h!]
\centering
\caption{Definitions}
\begin{tabular}{| C{6} | p{6cm} |}\hline
	\textbf{Term}	&\textbf{\centering Meaning}\\\hline
	X-axis					&Distance along the length of the pool table\\\hline
	Y-axis					&Distance across the width of the pool table\\\hline
	Z-axis					&Height above the pool table\\\hline
	End-effector			&The end of the arm that will strike the cue ball\\\hline
	$\theta$				&Rotational angle of end-effector\\\hline
	Cue 					&End-effector\\\hline
	Personal Computer		&A laptop that will be used to run the more involved computational tasks such as visual recognition and the shot selection algorithm\\\hline
	Camera					&Some form of image capture device (e.g. a digital camera, smartphone with a camera, etc.)\\\hline
	Table State				&The current positions of all the balls on the table\\\hline
	Entity					&Classes that have a state, behaviour and identity (e.g. Book, Car, Person, etc.)\\\hline
	Boundary				&Classes that interact with users or external systems\\\hline
	Double					&Double-precision floating point numbers\\\hline
\end{tabular}
\label{tab:Definitions}
\end{table}
\subsubsection{Acronyms \& Abbreviations}
Table \ref{tab:Acronyms} lists the acronyms and abbreviations used in this document.
\begin{table}[h!]
\centering
\caption{Acronyms and Abbreviations}
\begin{tabular}{| p{6cm} | p{6cm} |}\hline
	\textbf{Acronym/Abbreviation}	&\textbf{Meaning}\\\hline
	VR								&Visual Recognition\\\hline
	PC								&Personal Computer\\\hline
	$\mu$C							&Micro-Controller\\\hline
	CRC								&Class Responsibility Collaboration\\\hline
\end{tabular}
\label{tab:Acronyms}
\end{table}
\newpage

\section{Detailed Class Diagram}~\\
\begin{center}
	\includegraphics[width=\textwidth]{DetailedClassDiagram.png}
\captionof{figure}{The system's detailed class diagram.}
\label{fig:detailedClassDiagram}
\end{center}
\newpage

\section{Module Guide}
This section discusses the various modules that this system is comprised of. The modules are divided based on which program they belong to. For each module, its responsibilities, secrets, MIS, and MID will be outlined.
\subsection{Camera Modules}
The following is the module contained within the Camera subsystem.
\subsubsection{EventHandler}
\textbf{Responsibilities}
\begin{itemize}
	\item[-] Listen for request from PC Communicator
	\item[-] Take a photo
	\item[-] Communicate photo to PC Communicator
\end{itemize}~\\
\textbf{Secrets}
\begin{itemize}
	\item[-] Picture-taking process
\end{itemize}~\\
\textbf{MIS}\\[2mm]
This module is an always-running program which executes the function of taking a picture and communicating it back to the requesting program. While this module is not performing the previous function, it will be listening for a request to be made.
\\\\
\textbf{MID}\\[2mm]
The state charts below provide a succinct depiction of this module's internal design.
\\\\
\subsection{PC VR Program Modules}
The following is the module contained within the PC VR subsystem.
\subsubsection{TableStateVR}
\textbf{Responsibilities}
\begin{itemize}
	\item[-] Read image from a file
	\item[-] Locate balls using VR
	\item[-] Determine ball identities
	\item[-] Write table state to a file
\end{itemize}~\\
\textbf{Secrets}
\begin{itemize}
	\item[-] Object detection algorithm
	\item[-] Ball identification algorithm
\end{itemize}~\\
\textbf{MIS}\\[2mm]
This module takes in an image of a pool table and analyzes it in order to return the locations and identities of each of the pool balls on the table.
\\\\
\textbf{MID}\\[2mm]
This module has 4 main steps. First it must read in the image from a predetermined file location. Then it locates the pool balls in that picture using a Visual Recognition object detection algorithm (supplied by a MATLAB library). Next, it will identify which ball is which according to comparing pixel colours within the detected objects to the colours of different pool balls using a LAB colour space. Lastly it will write these results to a file.\\~\\
The following pseudocode better describes this process:
\begin{lstlisting}
read image from file (path to file);
locate balls (image);

for every detected object:
	sample pixels within;
	
	while (pixels not similar to ball archetype AND arcge);
		increment archetype being checked;
		
	if (index out of range):
		raise unidentifiable ball error;
	else:
		results[index for this ball] = this object's location;

return results;
\end{lstlisting}~\\
\subsection{PC Controller Modules}
The following are the modules contained within the PC Controller subsystem.
\subsubsection{InferenceEngine}
\textbf{Responsibilities}
\begin{itemize}
	\item[-] Calculate the best shot to be made
\end{itemize}~\\
\textbf{Secrets}
\begin{itemize}
	\item[-] Algorithm to choose which shots to simulate
	\item[-] The computer's ball type (i.e. stripes or solids)
	\item[-] The rules of pool
\end{itemize}~\\
\textbf{MIS}\\[2mm]
This module allows for specification of a TableState through the use of a 2-D array of doubles. Using that TableState, the module will simulate various potential shots in order to determine an optimal one which is then accessible to other classes.
\\\\
\textbf{MID}\\[2mm]
This module iterates through shots that are to be simulated by SimulationInstances. In order to minimize computation, it only looks at shots that have a hope of directly hitting another legal ball (e.g. the eight ball cannot be struck first unless all of the player's other balls are already sunk). At the end of this computation, this module returns the specification of the optimal shot.\\~\\
The following pseudocode better describes this process:
\begin{lstlisting}
determine balls to shoot;

while (legal balls left to check){
	calculate smallest angle from cue ball to target ball;
	calculate largest angle from cue ball to target ball;
	
	for (all angles from smallest to largest angle){
		for (every power option){
			create SimulationInstance for this shot;
			
			while (simulation not finished){
				simulation.update();
			}
			
			update current shot's score;
			
			if (score of new shot > optimal shot score):
				update optimal shot;
		}		
		current angle += ANGULAR_STEP;
	}
}

return new Shot(calculated X, calculated Y, optimal shot's angle, optimal shot's power);
\end{lstlisting}
\newpage

\subsubsection{PCCommunicator}
\textbf{Responsibilities}
\begin{itemize}
	\item[-] Listens for request from $\mu$C
	\item[-] Sends confirmation of receipt to $\mu$C
	\item[-] Sends shot specification to $\mu$C
	\item[-] Listens for confirmation of receipt from $\mu$C
	\item[-] Sends image capture request to camera
	\item[-] Listens for response from camera
	\item[-] Initiate the PC VR program
	\item[-] Read table state from file
\end{itemize}~\\
\textbf{Secrets}
\begin{itemize}
	\item[-] Receipt confirmation message contents
	\item[-] Maximum time awaiting a response
\end{itemize}~\\
\textbf{MIS}\\[2mm]
This module is an always-running process which communicates with the various programs in this system while also providing control flow for the PC Controller program.
\\\\
\textbf{MID}\\[2mm]
The state charts below provide a succinct depiction of this module's internal design.
\\\\
\subsubsection{SimulationInstance}
\textbf{Responsibilities}
\begin{itemize}
	\item[-] Maintain the positions and velocities of the balls on the table at the current time step
	\item[-] Update the positions and velocities of the balls on the table after a time step
	\item[-] Keep track of whether there is still movement happening
	\item[-] Keep track of the scoring of the simulation (of a shot)
\end{itemize}~\\
\textbf{Secrets}
\begin{itemize}
	\item[-] The method of calculating a shot's score
	\item[-] Shot simulation algorithm
	\item[-] Physical constants
	\item[-] Simulation time step
\end{itemize}~\\
\textbf{MIS}\\[2mm]
This module handles the physics simulation involved with taking a shot while also scoring the shot according to various criteria. The state is updated for a new time step by calling the appropriate command. Once it returns \textit{false}, there is no further motion and the simulation is complete.
\\\\
\textbf{MID}\\[2mm]
This module handles performing a discrete time step simulation of a shot according to a shot and initial state of the table. In order to achieve this, it updates to the next snapshot of the simulation until every ball remaining on the table is stationary.\\~\\
First, the following pseudocode describes how this object is created:
\begin{lstlisting}
Given:
	An array of balls
	The x and y components of the cue ball's initial velocity
	
inMotion = true;
score = 0;

Initialize the velocities array;

velocities[Ball.CUE__NUM][0] = initial x component;
velocities[Ball.CUE__NUM][1] = initial y component;
\end{lstlisting}~\\\\
The following pseudocode assumes a very fine time step but may not not be feasible given the imposed time constraints (which will be tested empirically):
\begin{lstlisting}
for (all balls on table){
	update position according to velocity;
	update velocity according to friction;
}

for (all balls on table){
	if (in pocket AND sufficiently slow){
		if (cue ball):
			score reduced;
		else if (8 ball AND not shooting 8):
			score reduced;
		else if (wrong type):
			score reduced;
		else:
			score increased;
			
		set position to be off table;
		set velocity to 0;
	}else{
		for (current ball to last ball){
			if (collision):
				compute resulting velocities;
		}
}

for (all balls on table){
	if (velocity != 0){
		inMotion = true;
		return inMotion;
	}
}
\end{lstlisting}~\\\\
\subsection{$\mu$C Modules}
The following are the modules contained within the $\mu$C subsystem.
\subsubsection{Controller}
\textbf{Responsibilities}
\begin{itemize}
	\item[-] Control flow of program operation
	\item[-] Interrupt operation if cancel instruction is received
\end{itemize}~\\
\textbf{Secrets}
\begin{itemize}
	\item[-] Set movement positions (for ``move'' commands)
	\item[-] Instruction dispatch process
\end{itemize}~\\
\textbf{MIS}\\[2mm]
This module handles the control flow for the $\mu$C Program including determining appropriate movement and creation of interrupts when necessary. 
\\\\
\textbf{MID}\\[2mm]
The state charts below provide a succinct depiction of most of this module's internal design. The only other notable component in this module is how it selects where to move when a ``move'' command is received. In that event, it compares the current location to the two set locations at either end of the table. Whichever one is furthest is the one which the machine is moved towards. This is designed in a way such that the machine moves as far away as possible from where it was in the way of the user.
\\\\
\subsubsection{SensorMonitor}
\textbf{Responsibilities}
\begin{itemize}
	\item[-] Sensing signals from buttons
	\item[-] Sensing signals from calibration sensors
\end{itemize}~\\
\textbf{Secrets}
\begin{itemize}
	\item[-] The method of noticing signals
\end{itemize}~\\
\textbf{MIS}\\[2mm]
This module monitors the control signals coming from the buttons and calibration sensors and notifies the Controller when one of these sensors are activated.
\\\\
\textbf{MID}\\[2mm]
The state charts below provide a succinct depiction of this module's internal design.
\\\\
\subsubsection{ShotInterpreter}
\textbf{Responsibilities}
\begin{itemize}
	\item[-] Translate shot instruction into list of appropriate signals
	\item[-] Control translational motors
	\item[-] Control rotational motor
	\item[-] Control pneumatic end-effector
\end{itemize}~\\
\textbf{Secrets}
\begin{itemize}
	\item[-] Algorithm to determine appropriate movement
	\item[-] Method of transmitting signals to machine
\end{itemize}~\\
\textbf{MIS}\\[2mm]
This module uses a movement specification that it is provided in order to compute and generate the signals necessary to have the machine perform the required motion.
\\\\
\textbf{MID}\\[2mm]
The state charts below provide a succinct depiction of this module's internal design.
\\\\
\subsubsection{$\mu$CCommunicator}
\textbf{Responsibilities}
\begin{itemize}
	\item[-] Send shot calculation request to the PC Controller program
	\item[-] Receive confirmation of receipt from PC Controller program
	\item[-] Receive shot specification from PC Controller program
	\item[-] Send confirmation of receipt to PC Controller program 
\end{itemize}~\\
\textbf{Secrets}
\begin{itemize}
	\item[-] Receipt confirmation message contents
	\item[-] Maximum time awaiting a response
\end{itemize}~\\
\textbf{MIS}\\[2mm]
This module handles communicating with the PC Controller Program in order to compute the optimal shot to take.
\\\\
\textbf{MID}\\[2mm]
The state charts below provide a succinct depiction of this module's internal design.
\\\\

\section{Scheduling of Tasks}
The goal of this section is to outline the ordering and maximum allowable time frames of tasks in this program.
\subsection{Allocation of Time}
From the requirements document, there is only 90 seconds allowed between pressing a button and a shot being made. The most difficult computational step will be the shot simulations and so this section will deduce how much time the machine will have for that step.\\~\\
Firstly we must account for how long the physical machine would need in the worst case (moving all the way across both axes and rotating 180 degrees). To be fair, we will allocate 20 seconds to this operation.\\~\\
From here we can divvy up the remaining time among the various computational blocks. The smaller tasks such as communication, recognizing the button press, and other such operations will be allocated a total of 5 seconds. The process of object detection and identification will be given 15 seconds. This leaves 50 seconds to process the necessary simulations.
\newpage

\subsection{State Charts}
The following charts illustrate the lifecycle of all relevant classes in this system. This section is meant to depict a more isolated picture of how each class will operate.\\[15mm]
\begin{center}
	\includegraphics[width=0.6\textwidth]{ControllerStateChart.png}
\captionof{figure}{A state chart for the Controller class.}
\label{fig:ControllerStateChart}
\end{center}
\newpage

\begin{center}
	\includegraphics[width=0.8\textwidth]{ShotInterpreterStateChart.png}
\captionof{figure}{A state chart for the ShotInterpreter class.}
\label{fig:ShotInterpreterStateChart}
\end{center}
\begin{multicols}{2}
~\vfill
\begin{center}
	\includegraphics[width=0.4\textwidth]{SensorMonitorStateChart.png}
\captionof{figure}{A state chart for the SensorMonitor class.}
\label{fig:SensorMonitorStateChart}
\end{center}
~\vfill
\begin{center}
	\includegraphics[width=0.35\textwidth]{uCCommunicatorStateChart.png}
\captionof{figure}{A state chart for the $\mu$CCommunicator class.}
\label{fig:uCCommunicatorStateChart}
\end{center}
\end{multicols}
\newpage

~\vfill
\begin{center}
	\includegraphics[width=0.9\textwidth]{PCCommunicatorStateChart.png}
\captionof{figure}{A state chart for the PCCommunicator class.}
\label{fig:PCCommunicatorStateChart}
\end{center}
~\vfill
\newpage

~\vfill
\begin{center}
	\includegraphics[width=0.6\textwidth]{InferenceEngineStateChart.png}
\captionof{figure}{A state chart for the InferenceEngine class.}
\label{fig:InferenceEngineStateChart}
\end{center}
~\vfill
\newpage

\begin{multicols}{2}
\begin{center}
	\includegraphics[width=0.175\textwidth]{TableStateVRStateChart.png}
\captionof{figure}{A state chart for the TableStateVR class.}
\label{fig:TableStateVRStateChart}
\end{center}
\columnbreak
~\vspace{15mm}
\begin{center}
	\includegraphics[width=0.3\textwidth]{EventHandlerStateChart.png}
\captionof{figure}{A state chart for the EventHandler class.}
\label{fig:EventHandlerStateChart}
\end{center}
\end{multicols}
\newpage

\subsection{Sequence Diagrams}
The following are various sequence diagrams for different actions the system is required to perform. These diagrams are meant to provide better context for how the classes interact with each other to perform certain tasks.\vfill
\begin{center}
	\includegraphics[width=0.6\textwidth]{MoveSequenceDiagram.png}
\captionof{figure}{A sequence diagram for the ``move'' operation.}
\label{fig:MoveSequenceDiagram}
\end{center}
~\vfill
\begin{center}
	\includegraphics[width=0.6\textwidth]{cancelSequenceDiagram.png}
\captionof{figure}{A sequence diagram for the ``cancel'' operation.}
\label{fig:CancelSequenceDiagram}
\end{center}
~\vfill
\newpage

\begin{sidewaysfigure}
	\begin{center}
		\includegraphics[width=\textwidth]{TakeShotSequenceDiagram.png}
	\captionof{figure}{A sequence diagram for the ``take shot'' operation.}
	\label{fig:TakeShotSequenceDiagram}
	\end{center}
\end{sidewaysfigure}

\end{document}

%TODO double check for split sections

