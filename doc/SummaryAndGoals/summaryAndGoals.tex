\documentclass[titlepage]{article}
\usepackage[left=15mm,right=15mm,top=1in,bottom=1in]{geometry}
\usepackage{amsmath}
\usepackage{array}
\usepackage{ulem}

\newcolumntype{C}[1]{>{\centering\arraybackslash} m{#1cm}}
\date{}				%Removes the date from the title page.

\title{Autonomous Pool Playing Robot\\~\\\textbf{Project Summary and Goals}}
\author{
Ernest Selman\\selmae@mcmaster.ca\\1201291\\~\\\and
Eric Le Fort\\leforte@mcmaster.ca\\1308609\\~\\\and
Guy Meyer\\meyerg@mcmaster.ca\\1320231\\~\\\and
Andrew Danha\\danhaas@mcmaster.ca\\1223881\\~\\\and
Max Moore\\moorem8@mcmaster.ca\\1320009\\~\\\and
Derek Savery\\saverydj@mcmaster.ca\\1219142\\~\\
}



\begin{document}
\maketitle

\tableofcontents
\listoffigures

\vfill
\begin{figure}[!htbp]
\centering
\begin{tabular}{| C{3} | C{2} | C{5} | C{2.5} |}\hline
	Date			&Revision \#	&Comments						&Authors\\\hline
	DD/MM/YYYY		&0				&- Initial document creation	&Ernest Selman\newline Eric Le Fort\newline Guy Meyer\newline Andrew Danha\newline Max Moore\newline Derek Savery\\\hline
\end{tabular}
\caption{Revision History}
\end{figure}
\newpage


\section{Project Summary}
~\indent The aim of this project is to create an automated pool playing robot. This robot will be able to play pool against a human opponent for recreation or training purposes as determined by the user.\\

What follows is a breakdown of what the system will do step-by-step in more specific detail. To initiate the robot's turn, the user will press a button signifying that the system should begin. A camera will then be used to view the table and, using visual recognition algorithms, map out the positions of the pool balls. Our system will then determine the best angle at which to take a shot and how to move the equipment to that position. Once that is done, the robot will move into place, lining up the shot. Using a specialized, built-in pool cue, the robot will then make its shot.\\

Once it is the user's turn, the user will also be able to press a button to indicate to the robot that it needs to move in order to give room for the human player's shot. The robot will then move to a position that is out of the way.

\section{Success Criteria}
In order for the project to be considered a success, minimum criteria must be met. These minimums are that:
\begin{enumerate}
	\item 90\% of the time a straight shot will have the cue ball hit a ball that it chooses itself;
	\item 50\% of the time, the system should be able to sink the intended ball if it's a straight shot, and;
	\item Users of this system must not be placed at risk by the system at any time.
\end{enumerate}

\section{Mid-Level Goals}
Once success is achieved, these goals will be the immediate avenues toward improvement. These goals include:
\begin{enumerate}
	\item 95\% of the time, a straight shot will have the cue ball hit a ball that it chooses itself;
	\item 70\% of the time, a bank shot will have the cue ball hit the intended ball;
	\item 80\% of the time, the cue ball will not be sunk in a shot;
	\item 75\% of the time, the system should be able to sink the intended ball if it can with a straight shot;
	\item 40\% of the time, the system should be able to sink the intended ball with a bank shot when necessary, and;
	\item The finished project is polished to the point of being marketable.
\end{enumerate}

\section{High-Level Goals}
These goals are what will be going well above what we expect to be able to achieve. They will only be met if we can somehow complete all other goals well before deadline. These goals include:
\begin{enumerate}
	\item 98\% of the time, a straight shot will have the cue ball hit a ball that it chooses itself;
	\item 80\% of the time, a bank shot will have the cue ball hit the intended ball;
	\item 90\% of the time, the cue ball will not be sunk in a shot;
	\item 90\% of the time, the system should be able to sink the intended ball if it can with a straight shot;
	\item 65\% of the time, the system should be able to sink the intended ball with a bank shot when necessary;
	\item shooting the cue ball in such a way that it is placed strategically to make the next player's shot more difficult, and;
	\item being able to take advanced shots such as curving or putting front- or back-spin on the cue ball.
\end{enumerate}

%TODO


\end{document}